\documentclass[fontsize=12pt, BCOR=10mm, DIV=14, parskip=true,headings=small]{scrartcl}
\usepackage[USenglish]{babel}
\usepackage{hyperref}
\usepackage[utf8]{inputenc}
\usepackage{listings}
\usepackage{microtype}
\usepackage{url}

\RedeclareSectionCommand[beforeskip=0.25\baselineskip, afterskip=.25\baselineskip]{section}

\lstset{basicstyle=\ttfamily, columns=fullflexible, keepspaces=true}

\title{Installation of the Geoparser Framework}
\author{Ludwig Richter (ludwig.richter@stud.uni-heidelberg.de)}
\date{\today}

\begin{document}

\maketitle

In this document, the installation of the Geoparser Framework is described. In the following it is assumed that you are using Eclipse as IDE.

\section{Install Java 8 JDK}

The geoparser software requires Java 8. Install the latest Java 8 JDK from \url{http://www.oracle.com/technetwork/pt/java/javase/downloads/index.html}
to be able to compile the source code.

\section{Install Eclipse and Plugins}

\subsection{Eclipse}
Install the latest \texttt{Eclipse IDE for Java Developers} from \url{http://www.eclipse.org/downloads/packages/}.

\subsection{Configure Java Compiler}
Ensure that the Java 8 JDK is used as JRE in Eclipse. Check that it is listed in \texttt{Window|\-Preferences|Java|Installed JREs}. If not, add its installation path via \texttt{Add...}. 

\subsection{SVN Plugin}
If you get the source code via SVN, an extra plugin is required in Eclipse. Install \texttt{Subversive - SVN Team Provider} via \texttt{Help\-|Eclipse\- Marketplace...}. When you import a project via SVN the first time, you will be prompted to select a \texttt{Subversive Connector} implementation. \texttt{SVN Kit} works just fine.

\subsection{JavaFX Plugin}
Install the plugin \texttt{e(fx)clipse}  via \texttt{Help|Eclipse Marketplace...} to enable JavaFX support within Eclipse.

\section{Other Optional Plugins}
Other useful plugins for programming may be \texttt{JSON Editor Plugin}, \texttt{FindBugs Eclipse Plugin}, and \texttt{EclEmma Java Code Coverage}. All of them can be installed via \texttt{Help\-|Eclipse Marketplace...}. I used them while developing the framework...

\subsection{SceneBuilder}
The installation of this tool is optional. If you want to implement JavaFX applications (i.e., programs with GUI), you may want to install SceneBuilder from \url{http://gluonhq.com/labs/scene-builder/#download}. This tool is a comfortable JavaFX editor that works together with Eclipse. In Eclipse, go to \texttt{Window|Preferences|JavaFX} and set the path to the executable of your SceneBuilder installation. Afterwards, you can open \texttt{fxml}-files by right-clicking on them in the Package Explorer view and selecting \texttt{Open With Scenebuilder}.

Note: for the viewer apps, the third-party \texttt{controlsfx} package is used. In SceneBuilder, you can add the library by clicking on the configuration button next to the Library search field. Click \texttt{JAR/FXML Manager}. Add the package by clicking on \texttt{Search repositories} and installing the latest version.

Note: currently, SceneBuilder is still not able to open files that already contain \texttt{controlsfx} controls (it is possible to add new ones, though). You need to manually comment them out beforehand... Seems to be a bug.

\section{Download Source Code}
If you got the source code directly as an archive or project, you can import the project via \texttt{File|Import...|Archive} or \texttt{File|Import|Existing Projects into Workspace}, respectively.

Otherwise, the source code for the Geoparser-Framework currently resides in \url{svn+ssh://username@svn-dbs.ifi.uni-heidelberg.de/students/sp/lrichter/code/GeoParser/trunk}. In Eclipse, you can simply import the project via \texttt{File|Import...|Project from SVN}. Ask the DBS administrators\footnote{dbs-admin@informatik.uni-heidelberg.de} for access rights.

\section{Install GeoNames Database}
In order to fill the gazetteer with data, you need to load a GeoNames dump into a PostgreSQL database. In the source code, see \url{/GeoParser/docu/geonames_installation.pdf} for instructions on how to populate the GeoNames database in PostgreSQL.

\section{Create Gazetteer Database}
You may either use the same PostgreSQL server, where you created the Geonames database, or a separate PostgreSQL server. It does not matter...
Create a database called \texttt{gazetteer}. Install the following required extensions afterwards:

\begin{lstlisting}
CREATE EXTENSION IF NOT EXISTS postgis;
CREATE EXTENSION IF NOT EXISTS fuzzystrmatch;
\end{lstlisting}

Note: the table schema will be automatically be generated by the geoparser via JPA.

\section{Configure Geoparser}
The Geoparser Framework can be configured by editing the file: \url{/GeoParser/src/main/resources/geoparser.config.json}. Alternatively, you can use \url{/GeoParser/src/main/java/org/unihd/dbs/geoparser/GeoParserConfig.java} to edit the configuration file pragmatically. The latter class also contains hints on how to format the file.

Create/modify the \texttt{dbConnectionConfigurations} entries that contain the connection details to GeoNames and the gazetteer, e.g.:

\begin{lstlisting}
"local.geonames": {
    "dbConnectionData": {
        "dbName": "geonames",
        "host": "localhost",
        "port": 5432,
        "userName": "postgres",
        "password": ["S", "O", "M", "E", "_", "P", "A", "S", "S", "W"],
        "authenticationRequired": true
    },
    "sshConnectionData": {
        "sshRequired": false
    }
},
"local.gazetteer": {
    "dbConnectionData": {
        "dbName": "gazetteer",
        "host": "localhost",
        "port": 5432,
        "userName": "postgres",
        "password": ["S", "O", "M", "E", "_", "P", "A", "S", "S", "W"],
        "authenticationRequired": true
    },
    "sshConnectionData": {
        "sshRequired": false
    }
}
\end{lstlisting}

Now, add/modify the following \texttt{configurationStrings} entry:

\begin{lstlisting}
"geonames.source": "local.geonames"
"gazetteer.persistence_unit.db_source": "local.gazetteer"
\end{lstlisting}

For experts only: The property \verb|gazetteer.persistence_unit.name| contains the label for the JPA persistence unit to be used. Persistence units can be modified in \url{/GeoParser/src/main/resources/META-INF/persistence.xml}.

\section{Populate Gazetteer Database}

Run the program \url{/GeoParser/src/main/java/org/unihd/dbs/geoparser/gazetteer/importers/GazetteerInstaller.java}. This program first creates the gazetteer table schema. 

WARNING: BE AWARE THAT THIS WILL DELETE ANY EXISTING GAZETTEER DATA IN THE DATABASE!!! 

To increase bulk-insertion performance, all foreign-key constraints and indexes are deleted. Next, all types defined in \url{de.unihd.dbs.geoparser.gazetteer.types} are imported. Then, the GeoNames data are imported. Finally, all foreign-key constraints and indexes are restored after all data have been successfully inserted.

Note: This task will take several hours to complete! On a i7-4700MQ @2.4GHz with 16GB RAM it took 3.5 hours. Ensure that you have 4GB available memory and 11-12GB of free data storage for the PostgreSQL database!

\section{Check if Everything Works}

You now should be able to view and search the gazetteer data using the \texttt{Gazetteer Viewer} provided in \url{/GeoParser/src/main/java/org/unihd/dbs/geoparser/gazetteer/viewer/GazetteerViewerApp.java}

\end{document}
